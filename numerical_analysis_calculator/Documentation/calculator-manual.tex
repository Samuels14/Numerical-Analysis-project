\documentclass[11pt,a4paper]{article}
\usepackage[utf8]{inputenc}
\usepackage[english]{babel}
\usepackage{amsmath}
\usepackage{geometry}
\usepackage{listings}
\usepackage{xcolor}
\usepackage{hyperref}
\usepackage{booktabs}
\usepackage{longtable}

\geometry{left=2.5cm,right=2.5cm,top=2.5cm,bottom=2.5cm}

% Configure listings for code examples
\lstdefinestyle{code}{
    basicstyle=\ttfamily\small,
    commentstyle=\color{gray}\itshape,
    keywordstyle=\bfseries\color{blue},
    numberstyle=\tiny\color{gray},
    stringstyle=\color{red},
    breaklines=true,
    breakatwhitespace=true,
    showstringspaces=false,
    frame=single,
    xleftmargin=1em,
    columns=flexible,
    keepspaces=true,
    language=Python
}

% Hyperref configuration
\hypersetup{
    colorlinks=true,
    linkcolor=blue,
    urlcolor=blue,
    citecolor=blue
}

\title{\textbf{User Manual\\Numerical Analysis Calculator}}
\author{}
\date{November 2025}

\begin{document}

\maketitle
\newpage

\tableofcontents
\newpage

%=============================================================================
\section{Introduction}
%=============================================================================

\subsection{What is the Numerical Analysis Calculator?}

The Numerical Analysis Calculator is a professional web application designed to solve complex mathematical problems using numerical methods. It implements \textbf{26 different methods} organized into three main categories:

\begin{itemize}
    \item \textbf{Nonlinear Equations} (7 methods)
    \item \textbf{Linear Systems} (13 methods)
    \item \textbf{Interpolation} (6 methods)
\end{itemize}

\subsection{Main Features}

\begin{itemize}
    \item[\checkmark] \textbf{Intuitive Interface}: Modern design inspired by professional calculators
    \item[\checkmark] \textbf{Interactive Graphics}: Function visualization with Desmos API
    \item[\checkmark] \textbf{Detailed Results}: Step-by-step iteration tables
    \item[\checkmark] \textbf{Responsive}: Works on PC, tablets and smartphones
    \item[\checkmark] \textbf{No Installation}: 100\% online web application
    \item[\checkmark] \textbf{Intelligent Validation}: Educational error messages
\end{itemize}

\subsection{Who is This Application For?}

\begin{itemize}
    \item Engineering and exact sciences students
    \item Mathematics and numerical analysis professors
    \item Engineers who need to solve numerical problems
    \item Researchers in scientific areas
    \item Anyone interested in numerical methods
\end{itemize}

\newpage
%=============================================================================
\section{Application Access}
%=============================================================================

\subsection{System Requirements}

\textbf{Supported Browsers:}
\begin{itemize}
    \item Google Chrome 90+ (Recommended)
    \item Mozilla Firefox 88+
    \item Safari 14+
    \item Microsoft Edge 90+
\end{itemize}

\textbf{Internet Connection:}
\begin{itemize}
    \item Required to load the application and Desmos graphics
\end{itemize}

\textbf{Devices:}
\begin{itemize}
    \item Desktop computers (Windows, Mac, Linux)
    \item Laptops
    \item Tablets (iPad, Android)
    \item Smartphones (iOS, Android)
\end{itemize}

\subsection{How to Access}

\begin{enumerate}
    \item Open your web browser
    \item Enter the application URL
    \item The application will load automatically
    \item No registration or login required
\end{enumerate}

\newpage
%=============================================================================
\section{User Interface}
%=============================================================================

\subsection{Main Components}

\subsubsection{Header}
\begin{itemize}
    \item \textbf{Logo}: $\Sigma$ (Sigma) identifying symbol
    \item \textbf{Title}: ``Numerical Calculator''
    \item \textbf{Description}: Information about the application
    \item \textbf{Available Functions}: List of supported mathematical functions
\end{itemize}

\subsubsection{Methods Panel (Left)}
Contains four main tabs:
\begin{itemize}
    \item \textbf{Equation Solving}: Equation methods
    \item \textbf{Linear Systems}: Linear equation systems
    \item \textbf{Interpolation}: Interpolation methods
    \item \textbf{Additional Methods}: Additional methods
\end{itemize}

\subsubsection{Results Panel (Right)}
\begin{itemize}
    \item \textbf{Graph}: Function visualization (when applicable)
    \item \textbf{Results}: Detailed output with iteration tables
    \item \textbf{Warnings}: Suggestions and informative messages
\end{itemize}

\subsection{Tab Navigation}

\textbf{To switch between categories:}
\begin{enumerate}
    \item Click on the desired tab
    \item Available methods will be displayed automatically
    \item Select a method by clicking on its card
\end{enumerate}

\textbf{Visual indicators:}
\begin{itemize}
    \item Active tab: Blue background and bottom line
    \item Selected method: Blue border and highlighted background
\end{itemize}

\subsection{Parameter Form}

Once a method is selected, a dynamic form will appear with the necessary parameters:

\begin{itemize}
    \item \textbf{Text fields}: For mathematical functions
    \item \textbf{Numeric fields}: For initial values, tolerances, etc.
    \item \textbf{Text areas}: For matrices and vectors
    \item \textbf{Calculate button}: Executes the selected method
\end{itemize}

\newpage
%=============================================================================
\section{Nonlinear Equation Methods}
%=============================================================================

Nonlinear equation methods are used to find roots (values of $x$ where $f(x) = 0$) of functions.

\subsection{Incremental Search}

\textbf{Description:}\\
Searches for intervals where the function changes sign, indicating the presence of a root.

\textbf{When to Use It:}
\begin{itemize}
    \item To explore a function and find all its roots
    \item When you don't know where the roots are
    \item As a preliminary step to other methods
\end{itemize}

\textbf{Parameters:}
\begin{itemize}
    \item \textbf{Function $f(x)$}: Function to analyze
    \item \textbf{$x_0$}: Starting search point
    \item \textbf{$\Delta$ (Delta)}: Increment (can be positive or negative)
    \item \textbf{Nmax}: Maximum number of iterations
\end{itemize}

\textbf{Usage Example:}
\begin{lstlisting}[style=code]
Function: x**3 - x - 2
x0: 0
Delta: 0.5
Nmax: 100
\end{lstlisting}

\textbf{Results:}
\begin{itemize}
    \item List of intervals $[a, b]$ where there is a sign change
    \item Table with values of $x$ and $f(x)$ at each iteration
\end{itemize}

\textbf{Advantages:}
\begin{itemize}
    \item Simple to understand
    \item Finds multiple roots
    \item Does not require derivatives
\end{itemize}

\textbf{Disadvantages:}
\begin{itemize}
    \item Does not find multiple roots (where $f(x)$ touches the axis but does not cross)
    \item Can be slow
    \item Depends on the value of $\Delta$
\end{itemize}

\newpage
\subsection{Bisection}

\textbf{Description:}\\
Repeatedly divides an interval in half until finding the root with desired precision.

\textbf{When to Use It:}
\begin{itemize}
    \item When you have an interval $[a, b]$ where $f(a)$ and $f(b)$ have opposite signs
    \item You need guaranteed convergence
    \item The function is continuous
\end{itemize}

\textbf{Parameters:}
\begin{itemize}
    \item \textbf{Function $f(x)$}: Function to solve
    \item \textbf{$a$}: Left endpoint of interval
    \item \textbf{$b$}: Right endpoint of interval
    \item \textbf{Tolerance}: Desired precision (e.g., $10^{-6}$)
    \item \textbf{Nmax}: Maximum iterations
\end{itemize}

\textbf{Usage Example:}
\begin{lstlisting}[style=code]
Function: x**3 - x - 2
a: 1
b: 2
Tolerance: 0.000001
Nmax: 50
\end{lstlisting}

\textbf{Results:}
\begin{itemize}
    \item Iteration table with:
    \begin{itemize}
        \item iter: Iteration number
        \item $a$: Current left endpoint
        \item $x_m$: Midpoint
        \item $b$: Current right endpoint
        \item $f(x_m)$: Function value at $x_m$
        \item $E$: Absolute error
    \end{itemize}
    \item Final approximate root
\end{itemize}

\textbf{Advantages:}
\begin{itemize}
    \item Always converges (if there is a root in the interval)
    \item Very robust
    \item Easy to implement
\end{itemize}

\textbf{Disadvantages:}
\begin{itemize}
    \item Slow convergence
    \item Requires initial interval with sign change
    \item Does not work for multiple roots
\end{itemize}

\textbf{Interpretation:}
\begin{itemize}
    \item If $E <$ Tolerance: Root found with desired precision
    \item If reached Nmax: Increase Nmax or reduce tolerance
\end{itemize}

\newpage
\subsection{False Position}

\textbf{Description:}\\
Similar to bisection but uses linear interpolation instead of midpoint.

\textbf{When to Use It:}
\begin{itemize}
    \item When you have an interval with sign change
    \item You want faster convergence than bisection
    \item The function is approximately linear in the interval
\end{itemize}

\textbf{Parameters:}
\begin{itemize}
    \item \textbf{Function $f(x)$}: Target function
    \item \textbf{$a$}: Left endpoint
    \item \textbf{$b$}: Right endpoint
    \item \textbf{Tolerance}: Precision
    \item \textbf{Nmax}: Maximum iterations
\end{itemize}

\textbf{Usage Example:}
\begin{lstlisting}[style=code]
Function: cos(x) - x
a: 0
b: 1
Tolerance: 1e-7
Nmax: 100
\end{lstlisting}

\textbf{Results:}
\begin{itemize}
    \item Table similar to bisection
    \item The point $x_m$ is calculated with the formula:
    \[x_m = \frac{f(b) \cdot a - f(a) \cdot b}{f(b) - f(a)}\]
\end{itemize}

\textbf{Advantages:}
\begin{itemize}
    \item Faster than bisection in many cases
    \item Guarantees convergence
\end{itemize}

\textbf{Disadvantages:}
\begin{itemize}
    \item Can converge very slowly if $f(a)$ and $f(b)$ are very different
    \item Can ``stagnate'' at one endpoint
\end{itemize}

\subsection{Fixed Point}

\textbf{Description:}\\
Finds a value $x$ such that $x = g(x)$, transforming $f(x) = 0$ into $x = g(x)$.

\textbf{When to Use It:}
\begin{itemize}
    \item When you can rewrite $f(x) = 0$ as $x = g(x)$
    \item For equations that naturally have fixed-point form
\end{itemize}

\textbf{Parameters:}
\begin{itemize}
    \item \textbf{Function $f(x)$}: Original function (for graphing)
    \item \textbf{Function $g(x)$}: Fixed-point function ($x = g(x)$)
    \item \textbf{$x_0$}: Initial value
    \item \textbf{Tolerance}: Precision
    \item \textbf{Nmax}: Maximum iterations
\end{itemize}

\textbf{Usage Example:}
\begin{lstlisting}[style=code]
f(x): x**2 - 2
g(x): sqrt(2)  (or also: 2/x)
x0: 1.5
Tolerance: 1e-6
Nmax: 100
\end{lstlisting}

\textbf{How to Create $g(x)$:}

From $f(x) = x^2 - 2 = 0$:
\begin{itemize}
    \item Option 1: $x^2 = 2 \rightarrow x = \sqrt{2} \rightarrow g(x) = \sqrt{2}$
    \item Option 2: $x = \frac{2}{x} \rightarrow g(x) = \frac{2}{x}$
\end{itemize}

\textbf{Convergence Criterion:}\\
For convergence, it must satisfy: $|g'(x)| < 1$ near the root

\textbf{Results:}
\begin{itemize}
    \item Table with $x_i$, $g(x_i)$, $f(x_i)$ and error
    \item Approximate value of the root
\end{itemize}

\textbf{Advantages:}
\begin{itemize}
    \item Simple to implement
    \item Does not require derivatives
\end{itemize}

\textbf{Disadvantages:}
\begin{itemize}
    \item Does not always converge
    \item Critically depends on the choice of $g(x)$
    \item Can diverge with bad $x_0$
\end{itemize}

\newpage
\subsection{Newton-Raphson (Newton)}

\textbf{Description:}\\
Uses the tangent to the function to approximate the root. Quadratic convergence.

\textbf{When to Use It:}
\begin{itemize}
    \item When you have or can calculate $f'(x)$
    \item You need fast convergence
    \item You have a good initial estimate
\end{itemize}

\textbf{Parameters:}
\begin{itemize}
    \item \textbf{Function $f(x)$}: Target function
    \item \textbf{Derivative $f'(x)$}: Derivative of $f$ (optional, calculated numerically if not provided)
    \item \textbf{$x_0$}: Initial value
    \item \textbf{Tolerance}: Precision
    \item \textbf{Nmax}: Maximum iterations
\end{itemize}

\textbf{Formula:}
\[x_{n+1} = x_n - \frac{f(x_n)}{f'(x_n)}\]

\textbf{Usage Example:}
\begin{lstlisting}[style=code]
f(x): x**3 - 2*x - 5
f'(x): 3*x**2 - 2
x0: 2
Tolerance: 1e-10
Nmax: 50
\end{lstlisting}

\textbf{Results:}
\begin{itemize}
    \item Table with iter, $x_i$, $f(x_i)$, and error
    \item Typical convergence in 3-5 iterations
\end{itemize}

\textbf{Advantages:}
\begin{itemize}
    \item Quadratic convergence (very fast)
    \item Requires few iterations
\end{itemize}

\textbf{Disadvantages:}
\begin{itemize}
    \item Requires calculating $f'(x)$
    \item Can diverge if $x_0$ is poorly chosen
    \item Fails if $f'(x) = 0$
\end{itemize}

\textbf{Tips:}
\begin{itemize}
    \item Choose $x_0$ close to the root
    \item If you know $f'(x)$ analytically, enter it for better precision
    \item If it diverges, try another $x_0$ or use bisection first
\end{itemize}

\newpage
\subsection{Secant}

\textbf{Description:}\\
Approximation of Newton's method that does not require calculating derivatives.

\textbf{When to Use It:}
\begin{itemize}
    \item When calculating $f'(x)$ is difficult or expensive
    \item You want fast convergence without derivatives
    \item You have two initial values close to the root
\end{itemize}

\textbf{Parameters:}
\begin{itemize}
    \item \textbf{Function $f(x)$}: Target function
    \item \textbf{$x_0$}: First initial value
    \item \textbf{$x_1$}: Second initial value
    \item \textbf{Tolerance}: Precision
    \item \textbf{Nmax}: Maximum iterations
\end{itemize}

\textbf{Formula:}
\[x_{n+1} = x_n - f(x_n) \cdot \frac{x_n - x_{n-1}}{f(x_n) - f(x_{n-1})}\]

\textbf{Usage Example:}
\begin{lstlisting}[style=code]
f(x): exp(x) - 3*x
x0: 0
x1: 1
Tolerance: 1e-8
Nmax: 50
\end{lstlisting}

\textbf{Results:}
\begin{itemize}
    \item Iteration table
    \item Super-linear convergence (between linear and quadratic)
\end{itemize}

\textbf{Advantages:}
\begin{itemize}
    \item Does not require derivatives
    \item Faster than fixed point
    \item Super-linear convergence
\end{itemize}

\textbf{Disadvantages:}
\begin{itemize}
    \item Requires two initial values
    \item Can diverge if $x_0$ and $x_1$ are poorly chosen
    \item Slightly slower than Newton
\end{itemize}

\subsection{Newton Multiple Roots}

\textbf{Description:}\\
Variant of Newton's method designed for multiple roots (where $f(x) = f'(x) = 0$).

\textbf{When to Use It:}
\begin{itemize}
    \item When you suspect the root has multiplicity $> 1$
    \item Standard Newton converges very slowly
    \item You have $f(x)$, $f'(x)$ and $f''(x)$
\end{itemize}

\textbf{Parameters:}
\begin{itemize}
    \item \textbf{Function $f(x)$}: Target function
    \item \textbf{Derivative $f'(x)$}: First derivative
    \item \textbf{Second Derivative $f''(x)$}: Second derivative
    \item \textbf{$x_0$}: Initial value
    \item \textbf{Tolerance}: Precision
    \item \textbf{Nmax}: Maximum iterations
\end{itemize}

\textbf{Formula:}
\[x_{n+1} = x_n - \frac{f(x) \cdot f'(x)}{[f'(x)]^2 - f(x) \cdot f''(x)}\]

\textbf{Usage Example:}
\begin{lstlisting}[style=code]
f(x): (x - 2)**3
f'(x): 3*(x - 2)**2
f''(x): 6*(x - 2)
x0: 1.5
Tolerance: 1e-10
Nmax: 50
\end{lstlisting}

\textbf{Multiple Roots:}
\begin{itemize}
    \item Multiplicity 2: $f(x) = (x-a)^2$
    \item Multiplicity 3: $f(x) = (x-a)^3$
\end{itemize}

\textbf{Advantages:}
\begin{itemize}
    \item Quadratic convergence even for multiple roots
    \item Standard Newton only converges linearly for these roots
\end{itemize}

\textbf{Disadvantages:}
\begin{itemize}
    \item Requires calculating $f''(x)$
    \item More complex to implement
\end{itemize}

\newpage
%=============================================================================
\section{Linear System Methods}
%=============================================================================

Linear system methods solve systems of equations of the form $Ax = b$.

\subsection{Input Format}

\textbf{Matrix A (coefficients):}
\begin{lstlisting}[style=code]
4 -1 0
-1 4 -1
0 -1 4
\end{lstlisting}

\textbf{Vector b (independent terms):}
\begin{lstlisting}[style=code]
15 10 10
\end{lstlisting}

\textbf{Format:}
\begin{itemize}
    \item Each row on a line
    \item Numbers separated by spaces
    \item No commas or brackets
\end{itemize}

\subsection{Gaussian Elimination}

\textbf{Description:}\\
Converts the matrix to upper triangular form through elementary operations.

\textbf{When to Use It:}
\begin{itemize}
    \item For small to medium systems (up to $\sim$100 equations)
    \item When the matrix is not singular
    \item You need exact solution (without iterations)
\end{itemize}

\textbf{Parameters:}
\begin{itemize}
    \item \textbf{Matrix A}: $n \times n$ coefficient matrix
    \item \textbf{Vector b}: Independent terms vector
\end{itemize}

\textbf{Usage Example:}
\begin{lstlisting}[style=code]
Matrix A:
2 1 -1
-3 -1 2
-2 1 2

Vector b:
8 -11 -3
\end{lstlisting}

\textbf{Process:}
\begin{enumerate}
    \item \textbf{Forward elimination}: Creates zeros below the diagonal
    \item \textbf{Back substitution}: Solves from the last equation
\end{enumerate}

\textbf{Results:}
\begin{itemize}
    \item Elimination stages (intermediate matrices)
    \item Solution vector $x$
\end{itemize}

\textbf{Advantages:}
\begin{itemize}
    \item Direct method (non-iterative)
    \item Exact solution (within numerical precision)
\end{itemize}

\textbf{Disadvantages:}
\begin{itemize}
    \item Can fail if there is zero pivot
    \item Numerically unstable without pivoting
\end{itemize}

\newpage
\subsection{Partial Pivoting}

\textbf{Description:}\\
Gaussian Elimination with row interchange to improve stability.

\textbf{When to Use It:}
\begin{itemize}
    \item When simple Gaussian fails
    \item To improve numerical stability
    \item As standard method for general systems
\end{itemize}

\textbf{Difference with Simple Gaussian:}
\begin{itemize}
    \item At each step, searches for the element of largest absolute value in the column
    \item Exchanges rows to put that element on the diagonal
\end{itemize}

\textbf{Example:}
\begin{lstlisting}[style=code]
Matrix A:
0.0001 1
1 1

Vector b:
1 2
\end{lstlisting}

Without pivoting: Incorrect result\\
With pivoting: Correct result

\textbf{Advantages:}
\begin{itemize}
    \item More numerically stable
    \item Prevents division by zero
\end{itemize}

\textbf{Disadvantages:}
\begin{itemize}
    \item Slightly slower than simple Gaussian
\end{itemize}

\subsection{Total Pivoting}

\textbf{Description:}\\
Searches for the pivot in the entire submatrix (rows and columns).

\textbf{When to Use It:}
\begin{itemize}
    \item Maximum numerical stability required
    \item Very ill-conditioned matrices
    \item When partial pivoting is not enough
\end{itemize}

\textbf{Process:}
\begin{itemize}
    \item Searches for maximum element in the submatrix
    \item Exchanges rows AND columns
    \item More complex but more stable
\end{itemize}

\textbf{Advantages:}
\begin{itemize}
    \item Maximum numerical stability
\end{itemize}

\textbf{Disadvantages:}
\begin{itemize}
    \item Slower
    \item More complex (requires marking column permutations)
\end{itemize}

\newpage
\subsection{LU Decomposition}

\textbf{Description:}\\
Decomposes $A$ into product of two matrices: $A = LU$
\begin{itemize}
    \item $L$: Lower triangular
    \item $U$: Upper triangular
\end{itemize}

\textbf{When to Use It:}
\begin{itemize}
    \item You need to solve multiple systems with the same $A$
    \item Calculate determinants
    \item Matrix inversion
\end{itemize}

\textbf{Advantages:}
\begin{itemize}
    \item Efficient for multiple $b$ vectors
    \item Useful for additional calculations (determinant, inverse)
\end{itemize}

\textbf{Process:}
\begin{enumerate}
    \item Decompose $A = LU$
    \item Solve $Ly = b$ (forward substitution)
    \item Solve $Ux = y$ (back substitution)
\end{enumerate}

\textbf{Variants:}

\subsubsection{LU Simple}
\begin{itemize}
    \item Without pivoting
    \item Can fail if there is zero pivot
\end{itemize}

\subsubsection{LU Partial Pivot}
\begin{itemize}
    \item With partial pivoting: $PA = LU$
    \item More stable
\end{itemize}

\subsubsection{Crout}
\begin{itemize}
    \item $L$ has diagonal, $U$ has 1s on diagonal
    \item Useful for certain algorithms
\end{itemize}

\subsubsection{Doolittle}
\begin{itemize}
    \item $L$ has 1s on diagonal, $U$ has diagonal
    \item More common
\end{itemize}

\subsection{Cholesky Decomposition}

\textbf{Description:}\\
Special factorization for symmetric and positive definite matrices: $A = LL^T$

\textbf{When to Use It:}
\begin{itemize}
    \item $A$ is symmetric ($A_{ij} = A_{ji}$)
    \item $A$ is positive definite (all eigenvalues $> 0$)
    \item Common in: least squares, finite elements
\end{itemize}

\textbf{Advantages:}
\begin{itemize}
    \item Faster than LU (half the operations)
    \item More numerically stable
    \item Requires less memory
\end{itemize}

\textbf{Disadvantages:}
\begin{itemize}
    \item Only for symmetric and positive definite matrices
\end{itemize}

\textbf{Valid Matrix Example:}
\begin{lstlisting}[style=code]
4 2 1
2 5 3
1 3 6
\end{lstlisting}

\textbf{How to Verify if Positive Definite:}
\begin{itemize}
    \item All principal minors are positive
    \item Or: All eigenvalues are positive
\end{itemize}

\newpage
\subsection{Iterative Methods}

Iterative methods start from an initial solution $x_0$ and refine it.

\subsubsection{Jacobi}

\textbf{Description:}\\
Solves each variable in terms of the others and updates simultaneously.

\textbf{When to Use It:}
\begin{itemize}
    \item Large and sparse matrices
    \item Diagonally dominant
    \item Parallelization
\end{itemize}

\textbf{Additional Parameters:}
\begin{itemize}
    \item \textbf{$x_0$ vector}: Initial solution (optional, default zeros)
    \item \textbf{Tolerance}: Precision
    \item \textbf{Nmax}: Maximum iterations
\end{itemize}

\textbf{Convergence Criterion:}
\begin{itemize}
    \item Spectral radius of $T < 1$
    \item Diagonally dominant (sufficient but not necessary)
\end{itemize}

\textbf{Diagonally Dominant:}
\[|a_{ii}| > \sum_{j \neq i} |a_{ij}|\]

\textbf{Example:}
\begin{lstlisting}[style=code]
Matrix A:
4 -1 0
-1 4 -1
0 -1 4

Vector b:
15 10 10

x0: (optional)
Tolerance: 1e-6
Nmax: 100
\end{lstlisting}

\textbf{Results:}
\begin{itemize}
    \item Iteration matrix $T$
    \item Vector $C$
    \item Spectral radius
    \item Iteration table
    \item Warning if may diverge
\end{itemize}

\subsubsection{Gauss-Seidel}

\textbf{Description:}\\
Similar to Jacobi but uses updated values immediately.

\textbf{When to Use It:}
\begin{itemize}
    \item Same type of matrices as Jacobi
    \item Typically converges faster than Jacobi
\end{itemize}

\textbf{Difference with Jacobi:}
\begin{itemize}
    \item Jacobi: Updates all variables simultaneously
    \item Gauss-Seidel: Updates variables sequentially using new values
\end{itemize}

\textbf{Advantages over Jacobi:}
\begin{itemize}
    \item Faster convergence (typically)
    \item Requires less memory
\end{itemize}

\textbf{Disadvantages:}
\begin{itemize}
    \item Not parallelizable
\end{itemize}

\newpage
\subsubsection{SOR (Successive Over-Relaxation)}

\textbf{Description:}\\
Gauss-Seidel with relaxation factor $\omega$ to accelerate convergence.

\textbf{Additional Parameters:}
\begin{itemize}
    \item \textbf{omega ($\omega$)}: Relaxation factor ($0 < \omega < 2$)
    \begin{itemize}
        \item $\omega < 1$: Under-relaxation (more stable)
        \item $\omega = 1$: Standard Gauss-Seidel
        \item $\omega > 1$: Over-relaxation (faster)
    \end{itemize}
\end{itemize}

\textbf{How to Choose $\omega$:}
\begin{itemize}
    \item Theoretical: Optimal $\omega$ depends on spectral radius of Gauss-Seidel
    \item Practical: Try values between 1.0 and 1.9
    \item Typical: 1.5 is good starting point
\end{itemize}

\textbf{Example:}
\begin{lstlisting}[style=code]
omega: 1.5
\end{lstlisting}

\textbf{Advantages:}
\begin{itemize}
    \item Can be much faster than Gauss-Seidel
    \item Useful for large matrices
\end{itemize}

\textbf{Disadvantages:}
\begin{itemize}
    \item Choosing optimal $\omega$ can be difficult
\end{itemize}

\subsection{Forward/Backward Substitution}

\textbf{Description:}\\
Direct methods for triangular systems.

\subsubsection{Forward Substitution}

\textbf{When to Use It:}
\begin{itemize}
    \item Lower triangular system $Lx = b$
    \item As step in LU factorization
\end{itemize}

\textbf{Input format:}\\
Augmented matrix $[L|b]$:
\begin{lstlisting}[style=code]
2 0 0 4
1 3 0 5
2 1 4 6
\end{lstlisting}

\subsubsection{Backward Substitution}

\textbf{When to Use It:}
\begin{itemize}
    \item Upper triangular system $Ux = b$
    \item As step in LU factorization or Gaussian
\end{itemize}

\textbf{Input format:}\\
Augmented matrix $[U|b]$:
\begin{lstlisting}[style=code]
2 1 -1 8
0 3 2 -11
0 0 4 -3
\end{lstlisting}

\newpage
%=============================================================================
\section{Interpolation Methods}
%=============================================================================

Interpolation constructs a function that passes through a given set of points.

\subsection{Input Format}

\textbf{X Points ($x$ coordinates):}
\begin{lstlisting}[style=code]
0 1 2 3 4
\end{lstlisting}

\textbf{Y Points ($y$ coordinates):}
\begin{lstlisting}[style=code]
1 2.5 3.2 4.1 5.5
\end{lstlisting}

\textbf{Requirements:}
\begin{itemize}
    \item Same number of values in X and Y
    \item Minimum 2 points
    \item Points will be automatically sorted by X
\end{itemize}

\subsection{Vandermonde}

\textbf{Description:}\\
Constructs a polynomial $P(x)$ of degree $n-1$ that passes through $n$ points.

\textbf{When to Use It:}
\begin{itemize}
    \item Few points ($< 10$)
    \item You need the explicit polynomial
    \item Well-spaced points
\end{itemize}

\textbf{Method:}\\
Solves system $Vc = y$ where $V$ is the Vandermonde matrix

\textbf{Example:}
\begin{lstlisting}[style=code]
X: 0 1 2
Y: 1 3 2
\end{lstlisting}

\textbf{Results:}
\begin{itemize}
    \item Vandermonde matrix
    \item Polynomial coefficients $[a_0, a_1, a_2, \ldots]$
    \item Polynomial: $P(x) = a_0x^n + a_1x^{n-1} + \cdots + a_n$
\end{itemize}

\textbf{Advantages:}
\begin{itemize}
    \item Obtains explicit polynomial
    \item Simple to understand
\end{itemize}

\textbf{Disadvantages:}
\begin{itemize}
    \item Numerically unstable for many points
    \item Runge phenomenon (oscillations) with many points
\end{itemize}

\textbf{Warning:}\\
For more than 10 points, consider using splines

\newpage
\subsection{Newton (Divided Differences)}

\textbf{Description:}\\
Constructs polynomial using divided differences.

\textbf{When to Use It:}
\begin{itemize}
    \item You need to add points incrementally
    \item More stable than Vandermonde
    \item Efficient evaluation
\end{itemize}

\textbf{Method:}\\
Constructs divided differences table

\textbf{Example:}
\begin{lstlisting}[style=code]
X: 1 2 3 4
Y: 1 0 3 4
\end{lstlisting}

\textbf{Results:}
\begin{itemize}
    \item Divided differences table
    \item Newton coefficients $[c_0, c_1, c_2, \ldots]$
    \item Polynomial: $P(x) = c_0 + c_1(x-x_0) + c_2(x-x_0)(x-x_1) + \cdots$
\end{itemize}

\textbf{Advantages:}
\begin{itemize}
    \item More stable than Vandermonde
    \item Easy to add points
    \item Efficient evaluation
\end{itemize}

\textbf{Disadvantages:}
\begin{itemize}
    \item Polynomial in non-standard form
\end{itemize}

\subsection{Lagrange}

\textbf{Description:}\\
Constructs polynomial as linear combination of Lagrange basis polynomials.

\textbf{When to Use It:}
\begin{itemize}
    \item You need theoretical form of polynomial
    \item Mathematical analysis
    \item Evaluation at few points
\end{itemize}

\textbf{Method:}\\
\[P(x) = \sum y_i L_i(x)\]

where
\[L_i(x) = \prod_{j \neq i} \frac{(x-x_j)}{(x_i-x_j)}\]

\textbf{Example:}
\begin{lstlisting}[style=code]
X: 0 1 2
Y: 1 2 0
\end{lstlisting}

\textbf{Results:}
\begin{itemize}
    \item Basis polynomials $L_i(x)$
    \item Final combination
\end{itemize}

\textbf{Advantages:}
\begin{itemize}
    \item Mathematically elegant form
    \item Does not require solving systems
\end{itemize}

\textbf{Disadvantages:}
\begin{itemize}
    \item Expensive to evaluate
    \item Same Runge problem as Vandermonde
\end{itemize}

\newpage
\subsection{Splines}

Splines divide the interval into segments and use different polynomials in each.

\subsubsection{Linear Splines}

\textbf{Description:}\\
Joins points with straight lines.

\textbf{When to Use It:}
\begin{itemize}
    \item Data with abrupt changes
    \item Simplicity over smoothness
    \item Basic visualization
\end{itemize}

\textbf{Form:}\\
$S_i(x) = a_ix + b_i$ for $x \in [x_i, x_{i+1}]$

\textbf{Advantages:}
\begin{itemize}
    \item Very simple
    \item No oscillations
    \item Fast to calculate
\end{itemize}

\textbf{Disadvantages:}
\begin{itemize}
    \item Not smooth (discontinuity in derivatives)
\end{itemize}

\subsubsection{Quadratic Splines}

\textbf{Description:}\\
Parabolas that join points with continuity in first derivative.

\textbf{When to Use It:}
\begin{itemize}
    \item You need more smoothness than linear
    \item You don't need continuous second derivative
\end{itemize}

\textbf{Form:}\\
$S_i(x) = a_ix^2 + b_ix + c_i$

\textbf{Requirements:}
\begin{itemize}
    \item Minimum 3 points
\end{itemize}

\textbf{Advantages:}
\begin{itemize}
    \item Smooth ($C^1$ continuous)
    \item Fewer oscillations than single polynomial
\end{itemize}

\textbf{Disadvantages:}
\begin{itemize}
    \item Discontinuous second derivative
\end{itemize}

\subsubsection{Cubic Splines}

\textbf{Description:}\\
Cubic polynomials with continuity up to second derivative.

\textbf{When to Use It:}
\begin{itemize}
    \item Maximum smoothness
    \item Scientific data
    \item Animation and graphics
    \item Standard in professional interpolation
\end{itemize}

\textbf{Form:}\\
$S_i(x) = a_ix^3 + b_ix^2 + c_ix + d_i$

\textbf{Boundary Conditions (Natural Splines):}
\begin{itemize}
    \item $S''(x_0) = 0$
    \item $S''(x_n) = 0$
\end{itemize}

\textbf{Advantages:}
\begin{itemize}
    \item Maximum smoothness ($C^2$ continuous)
    \item Minimal oscillations
    \item Industry standard
\end{itemize}

\textbf{Disadvantages:}
\begin{itemize}
    \item More complex to calculate
\end{itemize}

\textbf{Spline Comparison:}
\begin{itemize}
    \item Linear: Simple, not smooth
    \item Quadratic: Medium smoothness
    \item Cubic: Maximum smoothness (recommended)
\end{itemize}

\newpage
%=============================================================================
\section{Mathematical Function Syntax}
%=============================================================================

The application uses \textbf{SymPy} to process mathematical functions.

\subsection{Basic Operators}

\begin{table}[h]
\centering
\begin{tabular}{lll}
\toprule
\textbf{Operator} & \textbf{Symbol} & \textbf{Example} \\
\midrule
Addition & \texttt{+} & \texttt{x + 3} \\
Subtraction & \texttt{-} & \texttt{x - 5} \\
Multiplication & \texttt{*} & \texttt{2*x} \\
Division & \texttt{/} & \texttt{x/2} \\
Power & \texttt{**} & \texttt{x**2} ($x^2$) \\
Parentheses & \texttt{()} & \texttt{(x + 1)*(x - 2)} \\
\bottomrule
\end{tabular}
\end{table}

\subsection{Mathematical Functions}

\subsubsection{Powers and Roots}
\begin{lstlisting}[style=code]
x**2          # x squared
x**3          # x cubed
x**0.5        # Square root (also: sqrt(x))
x**(1/3)      # Cube root
sqrt(x)       # Square root
\end{lstlisting}

\subsubsection{Exponentials and Logarithms}
\begin{lstlisting}[style=code]
exp(x)        # e^x
log(x)        # Natural logarithm (ln x)
ln(x)         # Natural logarithm (alias of log)
log(x, 10)    # Base 10 logarithm
log(x, 2)     # Base 2 logarithm
\end{lstlisting}

\subsubsection{Trigonometric}
\begin{lstlisting}[style=code]
sin(x)        # Sine
cos(x)        # Cosine
tan(x)        # Tangent
cot(x)        # Cotangent
sec(x)        # Secant
csc(x)        # Cosecant
\end{lstlisting}

\subsubsection{Inverse Trigonometric}
\begin{lstlisting}[style=code]
asin(x)       # Arcsine
acos(x)       # Arccosine
atan(x)       # Arctangent
\end{lstlisting}

\subsubsection{Hyperbolic}
\begin{lstlisting}[style=code]
sinh(x)       # Hyperbolic sine
cosh(x)       # Hyperbolic cosine
tanh(x)       # Hyperbolic tangent
\end{lstlisting}

\subsubsection{Other Functions}
\begin{lstlisting}[style=code]
abs(x)        # Absolute value |x|
sign(x)       # Sign of x (-1, 0, 1)
\end{lstlisting}

\subsection{Constants}

\begin{lstlisting}[style=code]
E             # Number e (2.71828...)
pi            # Number pi (3.14159...)
\end{lstlisting}

\subsection{Valid Function Examples}

\subsubsection{Polynomials}
\begin{lstlisting}[style=code]
x**3 - 2*x + 1
x**4 - 5*x**3 + 6*x**2 - x + 2
(x + 1)*(x - 2)*(x + 3)
\end{lstlisting}

\subsubsection{Trigonometric}
\begin{lstlisting}[style=code]
sin(x) - 0.5
cos(x)**2 + sin(x)**2 - 1
tan(x) - x
sin(2*x) - cos(x)
\end{lstlisting}

\subsubsection{Exponentials}
\begin{lstlisting}[style=code]
exp(x) - 3
exp(-x**2)
2**x - 8
\end{lstlisting}

\subsubsection{Logarithmic}
\begin{lstlisting}[style=code]
log(x) - 2
ln(x**2 + 1)
log(x, 10) - 1
\end{lstlisting}

\subsubsection{Combined}
\begin{lstlisting}[style=code]
x*exp(-x) - 0.1
sin(x)/x - 0.5
log(sin(x)**2 + 1) - (1/2)
x**2 * cos(x) - 1
exp(x) * sin(x) - 1
\end{lstlisting}

\subsection{Common Errors}

\textbf{Incorrect:}
\begin{lstlisting}[style=code]
2x            # Missing *
x^2           # Use **, not superscript
sen(x)        # Must be sin(x)
^             # Use **, not ^
x**2 + 1)     # Unbalanced parentheses
\end{lstlisting}

\textbf{Correct:}
\begin{lstlisting}[style=code]
2*x
x**2
sin(x)
x**2 + 1
\end{lstlisting}

\newpage
%=============================================================================
\section{Complete Practical Examples}
%=============================================================================

\subsection{Example: Find $\sqrt{2}$}

\textbf{Problem:} Find the square root of 2 using Newton-Raphson.

\textbf{Solution:}

\begin{enumerate}
    \item \textbf{Formulate:} $\sqrt{2}$ is root of $f(x) = x^2 - 2$

    \item \textbf{Parameters:}
    \begin{itemize}
        \item Function: \texttt{x**2 - 2}
        \item Derivative: \texttt{2*x}
        \item $x_0$: \texttt{1.5}
        \item Tolerance: \texttt{1e-10}
        \item Nmax: \texttt{20}
    \end{itemize}

    \item \textbf{Steps in the Calculator:}
    \begin{itemize}
        \item Select ``Equation Solving'' tab
        \item Click on ``Newton-Raphson''
        \item Enter function and derivative
        \item Click ``Calculate''
    \end{itemize}

    \item \textbf{Expected Result:}
    \begin{itemize}
        \item Root: 1.414213562373095...
        \item Converges in $\sim$5 iterations
    \end{itemize}
\end{enumerate}

\subsection{Example: System of 3 Equations}

\textbf{Problem:} Solve the system:
\begin{align*}
2x + y - z &= 8 \\
-3x - y + 2z &= -11 \\
-2x + y + 2z &= -3
\end{align*}

\textbf{Solution:}

\begin{enumerate}
    \item \textbf{Matrix Format:} $Ax = b$

    \item \textbf{Parameters:}
    \begin{itemize}
        \item Matrix A:
\begin{lstlisting}[style=code]
2 1 -1
-3 -1 2
-2 1 2
\end{lstlisting}
        \item Vector b: \texttt{8 -11 -3}
    \end{itemize}

    \item \textbf{Method:} Gaussian Elimination with Partial Pivoting

    \item \textbf{Steps:}
    \begin{itemize}
        \item ``Linear Systems'' tab
        \item Click ``Partial Pivot''
        \item Enter matrix and vector
        \item Calculate
    \end{itemize}

    \item \textbf{Result:} $x = [2, 3, -1]$

    \item \textbf{Verification:}
    \begin{itemize}
        \item $2(2) + 3 - (-1) = 8$ \checkmark
        \item $-3(2) - 3 + 2(-1) = -11$ \checkmark
        \item $-2(2) + 3 + 2(-1) = -3$ \checkmark
    \end{itemize}
\end{enumerate}

\newpage
\subsection{Example: Experimental Data Interpolation}

\textbf{Problem:} You have temperature vs time measurements:

\begin{table}[h]
\centering
\begin{tabular}{lccccc}
\toprule
\textbf{Time (min)} & 0 & 1 & 2 & 3 & 4 \\
\textbf{Temp (�C)} & 20 & 25 & 28 & 30 & 31 \\
\bottomrule
\end{tabular}
\end{table}

Construct a function that interpolates this data.

\textbf{Solution with Cubic Spline:}

\begin{enumerate}
    \item \textbf{Parameters:}
    \begin{itemize}
        \item X values: \texttt{0 1 2 3 4}
        \item Y values: \texttt{20 25 28 30 31}
    \end{itemize}

    \item \textbf{Method:} Cubic Spline

    \item \textbf{Result:}
    \begin{itemize}
        \item 4 cubic polynomials (one for each interval)
        \item Smooth function passing through all points
    \end{itemize}

    \item \textbf{Use:}
    \begin{itemize}
        \item Estimate temperature at $t = 2.5$ min
        \item Evaluate corresponding spline
    \end{itemize}
\end{enumerate}

\subsection{Example: Method Comparison}

\textbf{Problem:} Find root of $f(x) = x^3 - x - 2$ near $x = 1.5$

\textbf{Comparison:}

\begin{table}[h]
\centering
\begin{tabular}{lccc}
\toprule
\textbf{Method} & \textbf{Iterations} & \textbf{Precision} & \textbf{Comments} \\
\midrule
Bisection $[1, 2]$ & 20 & $10^{-6}$ & Slow but safe \\
False Position & 15 & $10^{-6}$ & Faster \\
Newton ($x_0=1.5$) & 5 & $10^{-10}$ & Very fast \\
Secant ($x_0=1, x_1=2$) & 6 & $10^{-10}$ & Fast without derivative \\
\bottomrule
\end{tabular}
\end{table}

\textbf{Conclusion:}
\begin{itemize}
    \item For guarantee: Bisection
    \item For speed with derivative: Newton
    \item For speed without derivative: Secant
\end{itemize}

\newpage
%=============================================================================
\section{Results Interpretation}
%=============================================================================

\subsection{Iteration Tables}

\subsubsection{For Equation Methods}

\textbf{Typical columns:}
\begin{itemize}
    \item \textbf{iter}: Iteration number
    \item \textbf{$x_i$}: Current approximation
    \item \textbf{$f(x_i)$}: Function value
    \item \textbf{E}: Absolute error $|x_i - x_{i-1}|$
\end{itemize}

\textbf{Interpretation:}
\begin{itemize}
    \item \textbf{E decreases}: Method converges \checkmark
    \item \textbf{E increases}: Method diverges $\times$
    \item \textbf{E stagnant}: Slow convergence or stagnation
\end{itemize}

\textbf{Stopping Criterion:}
\begin{itemize}
    \item $E <$ Tolerance: Solution found
    \item iter = Nmax: Increase Nmax or change method
\end{itemize}

\subsubsection{For Iterative Methods (Systems)}

\textbf{Typical columns:}
\begin{itemize}
    \item \textbf{iter}: Iteration
    \item \textbf{E}: Error (norm of $|x^{(n)} - x^{(n-1)}|$)
    \item \textbf{$x_0, x_1, x_2, \ldots$}: Solution components
\end{itemize}

\textbf{Interpretation:}
\begin{itemize}
    \item \textbf{Spectral radius $< 1$}: Method converges
    \item \textbf{Spectral radius $\geq 1$}: Method may diverge
\end{itemize}

\subsection{Warning Messages}

\subsubsection{``Function does not change sign''}
\begin{itemize}
    \item \textbf{Cause:} No root in interval $[a, b]$
    \item \textbf{Solution:} Use Incremental Search to find valid interval
\end{itemize}

\subsubsection{``Zero pivot encountered''}
\begin{itemize}
    \item \textbf{Cause:} Division by zero in elimination
    \item \textbf{Solution:} Use method with pivoting
\end{itemize}

\subsubsection{``Matrix is not positive definite''}
\begin{itemize}
    \item \textbf{Cause:} Cholesky requires positive definite matrix
    \item \textbf{Solution:} Use LU instead of Cholesky
\end{itemize}

\subsubsection{``Spectral radius $\geq 1$''}
\begin{itemize}
    \item \textbf{Cause:} Iterative method may not converge
    \item \textbf{Solution:}
    \begin{itemize}
        \item Verify diagonal dominance
        \item Adjust omega in SOR
        \item Use direct method
    \end{itemize}
\end{itemize}

\subsubsection{``Tolerance is very large''}
\begin{itemize}
    \item \textbf{Suggestion:} Result may be imprecise
    \item \textbf{Action:} Reduce tolerance for greater precision
\end{itemize}

\subsection{Graphics}

\textbf{Graph Elements:}
\begin{itemize}
    \item \textbf{Function $f(x)$}: Main curve
    \item \textbf{X axis}: $x$ values
    \item \textbf{Y axis}: $f(x)$ values
    \item \textbf{X-axis crossing points}: Roots of $f(x) = 0$
\end{itemize}

\textbf{Use:}
\begin{itemize}
    \item Visualize function behavior
    \item Identify number and location of roots
    \item Verify results
\end{itemize}

\textbf{Desmos Interactivity:}
\begin{itemize}
    \item Zoom: Mouse wheel
    \item Pan: Drag
    \item Reset: Click home icon
\end{itemize}

\newpage
%=============================================================================
\section{Use Cases}
%=============================================================================

\subsection{Engineering}

\subsubsection{Structural Analysis}
\begin{itemize}
    \item \textbf{System:} Equilibrium equations
    \item \textbf{Method:} Gaussian Elimination or LU
    \item \textbf{Example:} Calculate forces in truss
\end{itemize}

\subsubsection{Electrical Circuits}
\begin{itemize}
    \item \textbf{System:} Kirchhoff's laws
    \item \textbf{Method:} Iterative methods for large networks
    \item \textbf{Example:} Mesh currents
\end{itemize}

\subsubsection{Heat Transfer}
\begin{itemize}
    \item \textbf{Method:} Finite differences $\rightarrow$ Linear system
    \item \textbf{Solver:} Gauss-Seidel or SOR
    \item \textbf{Use:} Temperature distribution
\end{itemize}

\subsection{Sciences}

\subsubsection{Chemistry}
\begin{itemize}
    \item \textbf{Problem:} Chemical equilibrium
    \item \textbf{Method:} Newton-Raphson
    \item \textbf{Example:} Equilibrium constants
\end{itemize}

\subsubsection{Physics}
\begin{itemize}
    \item \textbf{Problem:} Trajectories, energies
    \item \textbf{Method:} Various depending on problem
    \item \textbf{Example:} Planetary orbits
\end{itemize}

\subsubsection{Biology}
\begin{itemize}
    \item \textbf{Problem:} Population models
    \item \textbf{Method:} Interpolation for experimental data
    \item \textbf{Example:} Population growth
\end{itemize}

\subsection{Finance}

\subsubsection{IRR Calculation (Internal Rate of Return)}
\begin{itemize}
    \item \textbf{Equation:} NPV = 0
    \item \textbf{Method:} Bisection or Newton
    \item \textbf{Use:} Evaluate investment projects
\end{itemize}

\subsubsection{Option Models}
\begin{itemize}
    \item \textbf{System:} Discretized Black-Scholes equations
    \item \textbf{Method:} Linear systems
    \item \textbf{Use:} Derivative valuation
\end{itemize}

\subsection{Education}

\subsubsection{Manual Calculation Verification}
\begin{itemize}
    \item \textbf{Use:} Check exercises
    \item \textbf{Advantage:} See complete iteration table
\end{itemize}

\subsubsection{Method Comparison}
\begin{itemize}
    \item \textbf{Use:} Understand differences
    \item \textbf{Example:} Why is Newton faster?
\end{itemize}

\subsubsection{Visualization}
\begin{itemize}
    \item \textbf{Use:} Graphics to understand concepts
    \item \textbf{Example:} See convergence graphically
\end{itemize}

\newpage
%=============================================================================
\section{Troubleshooting}
%=============================================================================

\subsection{Common Problems}

\subsubsection{``Application Error''}
\begin{itemize}
    \item \textbf{Cause:} Error in function syntax
    \item \textbf{Solution:}
    \begin{itemize}
        \item Verify syntax (use \texttt{*} for multiplication)
        \item Use parentheses correctly
        \item Consult syntax section
    \end{itemize}
\end{itemize}

\subsubsection{Incorrect Result}
\textbf{Possible causes:}
\begin{itemize}
    \item Incorrectly entered function
    \item Incorrect parameters
    \item Inappropriate method for the problem
\end{itemize}

\textbf{Verification:}
\begin{itemize}
    \item Evaluate $f(\text{root}) \approx 0$
    \item Substitute solution in original system
    \item Compare with manual calculation
\end{itemize}

\subsubsection{Does Not Converge}
\textbf{For Equations:}
\begin{itemize}
    \item Try better initial $x_0$
    \item Increase Nmax
    \item Change method
\end{itemize}

\textbf{For Iterative Systems:}
\begin{itemize}
    \item Verify diagonal dominance
    \item Adjust omega (SOR)
    \item Use direct method
\end{itemize}

\subsubsection{Graph Not Displayed}
\textbf{Causes:}
\begin{itemize}
    \item Lost internet connection
    \item Function with restricted domain
    \item Syntax error
\end{itemize}

\textbf{Solution:}
\begin{itemize}
    \item Verify connection
    \item Simplify function to test
    \item Reload page
\end{itemize}

\subsection{Best Practices}

\subsubsection{Choosing Appropriate Method}

\textbf{For Equations:}
\begin{enumerate}
    \item Do you have interval $[a, b]$? $\rightarrow$ Bisection/False Position
    \item Can you calculate $f'(x)$? $\rightarrow$ Newton
    \item Want speed without derivative? $\rightarrow$ Secant
    \item Explore roots? $\rightarrow$ Incremental Search first
\end{enumerate}

\textbf{For Linear Systems:}
\begin{enumerate}
    \item Small system ($<100$)? $\rightarrow$ Gaussian/LU
    \item Large and sparse system? $\rightarrow$ Iterative (Jacobi/GS/SOR)
    \item Symmetric positive definite matrix? $\rightarrow$ Cholesky
    \item Maximum stability? $\rightarrow$ Total Pivoting
\end{enumerate}

\textbf{For Interpolation:}
\begin{enumerate}
    \item Few points ($<10$)? $\rightarrow$ Polynomial (Newton/Lagrange)
    \item Many points? $\rightarrow$ Splines
    \item Maximum smoothness? $\rightarrow$ Cubic Spline
    \item Simplicity? $\rightarrow$ Linear Spline
\end{enumerate}

\subsubsection{Validate Results}

\textbf{Equations:}
\begin{quote}
Verify: $f(\text{root}) \approx 0$
\end{quote}

\textbf{Systems:}
\begin{quote}
Verify: $Ax \approx b$\\
Calculate: $\|Ax - b\|$
\end{quote}

\textbf{Interpolation:}
\begin{quote}
Verify: $P(x_i) = y_i$ for all points
\end{quote}

\subsubsection{Precision vs Cost}

\textbf{High Precision (Small Tolerance):}
\begin{itemize}
    \item More iterations
    \item More time
    \item Better result
\end{itemize}

\textbf{Low Precision (Large Tolerance):}
\begin{itemize}
    \item Fewer iterations
    \item Faster
    \item May be sufficient
\end{itemize}

\textbf{Recommendations:}
\begin{itemize}
    \item Engineering: $10^{-6}$ to $10^{-10}$
    \item Finance: $10^{-8}$
    \item Education: $10^{-6}$
\end{itemize}

\newpage
%=============================================================================
\section{Glossary}
%=============================================================================

\begin{description}
    \item[Convergence] Property that iterations approach the solution.

    \item[Linear Convergence] Error reduces by constant factor each iteration.

    \item[Quadratic Convergence] Error reduces to square each iteration (very fast).

    \item[Diagonally Dominant] Matrix where each diagonal element is greater than sum of others in its row.

    \item[Absolute Error] $|\text{approximate value} - \text{previous value}|$

    \item[Relative Error] $|\text{absolute error} / \text{approximate value}|$

    \item[Factorization] Decomposition of matrix into product of simpler matrices.

    \item[Interpolation] Construct function that passes through given points.

    \item[Iteration] One step of the numerical method.

    \item[Positive Definite Matrix] Symmetric matrix with all positive eigenvalues.

    \item[Singular Matrix] Non-invertible matrix (determinant = 0).

    \item[Sparse Matrix] Matrix with many zeros.

    \item[Symmetric Matrix] Matrix where $A_{ij} = A_{ji}$.

    \item[Triangular Matrix] Matrix with zeros above or below diagonal.

    \item[Nmax] Maximum number of iterations allowed.

    \item[Pivot] Diagonal element used in elimination.

    \item[Pivoting] Row/column exchange to improve stability.

    \item[Polynomial] Function of the form $a_0 + a_1x + a_2x^2 + \cdots$

    \item[Spectral Radius] Maximum absolute value of eigenvalues (determines convergence).

    \item[Root] Value $x$ where $f(x) = 0$.

    \item[Multiple Root] Root where $f(x) = f'(x) = 0$.

    \item[Linear System] Set of linear equations $Ax = b$.

    \item[Spline] Piecewise function (different polynomials in each interval).

    \item[Tolerance] Desired precision (stopping criterion).
\end{description}

\newpage
%=============================================================================
\section{References}
%=============================================================================

\subsection{Recommended Books}

\begin{enumerate}
    \item \textbf{Burden, R. L., \& Faires, J. D.} (2010). \textit{Numerical Analysis} (9th ed.). Brooks/Cole.

    \item \textbf{Chapra, S. C., \& Canale, R. P.} (2014). \textit{Numerical Methods for Engineers} (7th ed.). McGraw-Hill.

    \item \textbf{Press, W. H., et al.} (2007). \textit{Numerical Recipes: The Art of Scientific Computing} (3rd ed.). Cambridge University Press.

    \item \textbf{Kincaid, D., \& Cheney, W.} (2009). \textit{Numerical Analysis: Mathematics of Scientific Computing} (3rd ed.). American Mathematical Society.
\end{enumerate}

\subsection{Online Resources}

\begin{itemize}
    \item \textbf{Desmos Calculator}: \url{https://www.desmos.com/calculator}
    \item \textbf{SymPy Documentation}: \url{https://docs.sympy.org/}
    \item \textbf{NumPy Documentation}: \url{https://numpy.org/doc/}
\end{itemize}

\subsection{Mathematical Concepts}

\textbf{Intermediate Value Theorem:} If $f(a)$ and $f(b)$ have opposite signs, there exists a root in $[a, b]$.

\textbf{Banach Fixed Point Theorem:} If $|g'(x)| < 1$, then $x = g(x)$ converges.

\textbf{Sassenfeld Convergence Criterion:} For iterative methods.

\textbf{Diagonal Dominance Condition:} Sufficient for convergence of Jacobi and Gauss-Seidel.

\newpage
%=============================================================================
\section{Appendix: Quick Formulas}
%=============================================================================

\subsection*{Equation Methods}

\textbf{Bisection:}
\[x_m = \frac{a + b}{2}\]

\textbf{False Position:}
\[x_m = \frac{f(b) \cdot a - f(a) \cdot b}{f(b) - f(a)}\]

\textbf{Fixed Point:}
\[x_{n+1} = g(x_n)\]

\textbf{Newton:}
\[x_{n+1} = x_n - \frac{f(x_n)}{f'(x_n)}\]

\textbf{Secant:}
\[x_{n+1} = x_n - f(x_n) \cdot \frac{x_n - x_{n-1}}{f(x_n) - f(x_{n-1})}\]

\textbf{Newton Multiple Roots:}
\[x_{n+1} = x_n - \frac{f(x) \cdot f'(x)}{[f'(x)]^2 - f(x) \cdot f''(x)}\]

\subsection*{Iterative Methods}

\textbf{Jacobi:}
\[x^{(n+1)} = T \cdot x^{(n)} + C\]
where $T = -D^{-1}(L + U)$ and $C = D^{-1}b$

\textbf{Gauss-Seidel:}
\[T = -(D - L)^{-1}U, \quad C = (D - L)^{-1}b\]

\textbf{SOR:}
\[T = (D - \omega L)^{-1}[(1-\omega)D + \omega U], \quad C = \omega(D - \omega L)^{-1}b\]

\newpage
%=============================================================================
\section{Frequently Asked Questions (FAQ)}
%=============================================================================

\textbf{Q: Does the application save my calculations?}\\
A: No, the application does not save data. Each session is independent.

\vspace{0.5cm}

\textbf{Q: Can I export results?}\\
A: You can copy and paste the results, or use the browser's print function (Ctrl+P).

\vspace{0.5cm}

\textbf{Q: Does it work offline?}\\
A: No, it requires connection to load the application and graphics.

\vspace{0.5cm}

\textbf{Q: Is there a calculation limit?}\\
A: No limit. You can do all the calculations you need.

\vspace{0.5cm}

\textbf{Q: Can I trust the results for professional work?}\\
A: The methods are correct, but always verify critical results with other tools.

\vspace{0.5cm}

\textbf{Q: Why doesn't my method converge?}\\
A: Review the ``Troubleshooting'' section and verify that the parameters are adequate.

\vspace{0.5cm}

\textbf{Q: How do I report an error?}\\
A: Contact the site administrator with details of the problem and parameters used.

\newpage
%=============================================================================
\section{Contact and Support}
%=============================================================================

For questions, suggestions or to report problems:

\begin{itemize}
    \item \textbf{Email}: scadavidz@eafit.edu.co
    \item \textbf{Documentation}: This manual
\end{itemize}

\vspace{0.5cm}

\textbf{Date}: November 2025\\
\textbf{Language}: English

\vspace{2cm}

\begin{center}
\Large
\textbf{Thank you for using the Numerical Analysis Calculator!}
\end{center}

\end{document}
